% Copyright 2004 by Till Tantau <tantau@users.sourceforge.net>.
%
% In principle, this file can be redistributed and/or modified under
% the terms of the GNU Public License, version 2.
%
% However, this file is supposed to be a template to be modified
% for your own needs. For this reason, if you use this file as a
% template and not specifically distribute it as part of a another
% package/program, I grant the extra permission to freely copy and
% modify this file as you see fit and even to delete this copyright
% notice. 

\documentclass[aspectratio=169]{beamer}

\usepackage{listliketab}

\usepackage{../include/cs-seminar}
\usepackage{../include/lstcoq}
\usepackage{../include/lstisabelle}

\lstset{ 
  frame=none,
  captionpos=t, 
}

\usepackage{caption}
\DeclareCaptionFont{ninept}{\fontsize{8pt}{11pt}\selectfont #1}
\captionsetup{singlelinecheck=false, font=ninept}
\renewcommand*\lstlistingname{Example}  % not Example, not Listing. See

% There are many different themes available for Beamer. A comprehensive
% list with examples is given here:
% http://deic.uab.es/~iblanes/beamer_gallery/index_by_theme.html
% You can uncomment the themes below if you would like to use a different
% one:
%\usetheme{AnnArbor} % фу желто-синий
%\usetheme{Antibes} % черно-синий
%\usetheme{Bergen}
%\usetheme{Berkeley}
%\usetheme{Berlin}
%\usetheme{Boadilla}  %бирюз оч милая
%\usetheme{boxes}
%\usetheme{CambridgeUS} %красная красивая
%\usetheme{Copenhagen}
%\usetheme{Darmstadt}
%\usetheme{default}  %прост белая
%\usetheme{Frankfurt}
\usetheme{Goettingen} %голубая с оглавлением
%\usetheme{Hannover}
%\usetheme{Ilmenau}
%\usetheme{JuanLesPins}  %черная, ничего так
%\usetheme{Luebeck}
%\usetheme{Madrid} %стандартн
%\usetheme{Malmoe}
%\usetheme{Marburg}
%\usetheme{Montpellier} %светлая супеh-красивая
%\usetheme{PaloAlto}
%\usetheme{Pittsburgh}
%\usetheme{Rochester} %синяя квадратная
%\usetheme{Singapore}
%\usetheme{Szeged} % тоже милая
%\usetheme{Warsaw}

%\setbeamertemplate{itemize items}[circle]

\title{Comparison of two theorem provers: \\ Isabelle \& Coq}

% A subtitle is optional and this may be deleted
%\subtitle{Optional Subtitle}

%\author{A.~Yushkovskiy\inst{1} \and S.~Tripakis\inst{1}}
\author{A.~Yushkovskiy \and S.~Tripakis}
% - Give the names in the same order as the appear in the paper.
% - Use the \inst{?} command only if the authors have different
%   affiliation.

%\institute[Universities of Somewhere and Elsewhere] % (optional, but mostly needed)
\institute[AaltoUniversity] % (optional, but mostly needed)
{
  %\inst{1}%
  Department of Computer Science \\
  School of Science \\
  \textbf{Aalto University}
  %\and
  %\inst{2}%
  %Department of Theoretical Philosophy\\
  %University of Elsewhere
  }
  % - Use the \inst command only if there are several affiliations.
  % - Keep it simple, no one is interested in your street address.

\date{ $ $\\ CS-E4000: Seminar in Computer Science \\ autumn 2017 }
% - Either use conference name or its abbreviation.
% - Not really informative to the audience, more for people (including
%   yourself) who are reading the slides online

\subject{Theoretical Computer Science}
% This is only inserted into the PDF information catalog. Can be left
% out. 

% If you have a file called "university-logo-filename.xxx", where xxx
% is a graphic format that can be processed by latex or pdflatex,
% resp., then you can add a logo as follows:

% \pgfdeclareimage[height=0.5cm]{university-logo}{university-logo-filename}
% \logo{\pgfuseimage{university-logo}}

% Delete this, if you do not want the table of contents to pop up at
% the beginning of each subsection:
%\AtBeginSubsection[]
%{
%  \begin{frame}<beamer>{Outline}
%    \tableofcontents[currentsection,currentsubsection]
%  \end{frame}
%}

% Let's get started
\begin{document}

\begin{frame}
  \titlepage
\end{frame}

\begin{frame}{Outline}
  \tableofcontents
  % You might wish to add the option [pausesections]
\end{frame}

% Section and subsections will appear in the presentation overview
% and table of contents.
\section{Foundations of Formal Approach}

\subsection{The Formal System}

\begin{frame}{Definition of the Formal System}%{Optional Subtitle}
  \begin{itemize}
  \item {
    // TODO
  }
  \item {
    My second point.
  }
  \end{itemize}
\end{frame}

\subsection{Properties of a Formal System}
\begin{frame}{Properties of a Formal System}

A formal system $\Gamma = \ <A, V, \Omega, R>$ is called:

%\storestyleof{itemize}
%\begin{listliketab}
%\begin{tabular}{Llll}
\begin{itemize}
   \item  \textit{consistent}, \ \ \ \ if $\nexists \phi \in \Gamma: \ \Gamma \vdash \phi \land  \Gamma \vdash \neg \phi  \ \Leftrightarrow \ \Gamma \nvdash \bot$; \\
  \item \textit{complete}, \ \ \ \ \ if $\forall \phi \in U: \ A \vdash \phi \lor A \vdash \neg \phi$ ; \\
  \item \textit{independent}, \ if $\not \exists a \in A: \ A \vdash a$.
\end{itemize}
%\end{tabular}
%\end{listliketab}

\end{frame}

\subsection{Classical and Intuitionistic Logics}

% You can reveal the parts of a slide one at a time
% with the \pause command:
\begin{frame}{Classical and Intuitionistic Logics}
// Classical: set of axioms

// Intuitionistic: same, but without EM
\end{frame}



\section{Two Theorem Provers}

\subsection{Isabelle}

\begin{frame}[fragile]{Isabelle: First Acquaintance}

%Isabelle/HOL includes powerful specification tools, e.g. for (co)datatypes, (co)inductive definitions and recursive functions with complex pattern matching.

\vspace{10pt}
\begin{itemize}
  \item a generic proof assistant
  \item a successor of HOL theorem prover //TODO: cite
  \item created in 1986 by
    \begin{itemize}
    \item Larry Paulson \ @ University of Cambridge, and
    \item Tobias Nipkow @ Technische Universit\"{a}t M\"{u}nchen
    \end{itemize}
  \item based on classical higher-order logic
  \item uses powerful functional language \texttt{HOL}
  \item has large collection of formalised theories  //TODO: HOL, ZF, CCL, ...
\end{itemize}

\begin{tabular}{p{.45\linewidth} p{.45\linewidth}}
\begin{lstlisting}[language=isabelle, caption={Definition of basic datatypes 
%in Isabelle:
}]
datatype bool = 
  True | False
  
datatype nat = 
  zero ("0") | Suc nat
\end{lstlisting}
&
   
\begin{lstlisting}[language=isabelle,caption={???}]
???
\end{lstlisting}
\end{tabular}
%\end{raggedleft}

\end{frame}

\subsection{Coq}


\begin{frame}[fragile]{Coq: First Acquaintance}

\vspace{10pt}
\begin{itemize}
  \item a formal proof management system
  \item created at INRIA (Paris, France) in 1984
  \item based on Calculus of Inductive Constructions theory (an implementation of intuitionistic logic)
  \item uses powerful functional language \texttt{Gallina}
  \item has large collection of formalised theories  //TODO
  \item widely used in software verification (proof code extraction)
\end{itemize}

\begin{tabular}{p{.45\linewidth} p{.45\linewidth}}
  \begin{lstlisting}[language=coq, caption={Definition of basic datatypes 
  %in Isabelle:
  }]
Inductive False : Prop := .

Inductive True : Prop := I : True.

Inductive nat : Type :=
  | O : nat
  | S : nat -> nat.
\end{lstlisting}
  &
  
  \begin{lstlisting}[language=isabelle,caption={???}]
  ???
  \end{lstlisting}
\end{tabular}
%\end{raggedleft}

\end{frame}




\section{Comparison of the Theorem Provers}
\subsection{Common Features}
\subsection{Major Differences}

\begin{frame}{Blocks}
\begin{block}{Block Title}
You can also highlight sections of your presentation in a block, with it's own title
\end{block}
\begin{theorem}
There are separate environments for theorems, examples, definitions and proofs.
\end{theorem}
\begin{example}
Here is an example of an example block.
\end{example}
\end{frame}

% Placing a * after \section means it will not show in the
% outline or table of contents.
\section*{Summary}

\begin{frame}{Summary}
  \begin{itemize}
  \item
    The \alert{first main message} of your talk in one or two lines.
  \item
    The \alert{second main message} of your talk in one or two lines.
  \item
    Perhaps a \alert{third message}, but not more than that.
  \end{itemize}
  
  \begin{itemize}
  \item
    Outlook
    \begin{itemize}
    \item
      Something you haven't solved.
    \item
      Something else you haven't solved.
    \end{itemize}
  \end{itemize}
\end{frame}



% All of the following is optional and typically not needed. 
\appendix
\section<presentation>*{\appendixname}
\subsection<presentation>*{For Further Reading}

\begin{frame}[allowframebreaks]
  \frametitle<presentation>{For Further Reading}
    
  \begin{thebibliography}{10}
    
  \beamertemplatebookbibitems
  % Start with overview books.

  \bibitem{Author1990}
    A.~Author.
    \newblock {\em Handbook of Everything}.
    \newblock Some Press, 1990.
 
    
  \beamertemplatearticlebibitems
  % Followed by interesting articles. Keep the list short. 

  \bibitem{Someone2000}
    S.~Someone.
    \newblock On this and that.
    \newblock {\em Journal of This and That}, 2(1):50--100,
    2000.
  \end{thebibliography}
\end{frame}

\end{document}


