\documentclass[article]{aaltoseries}
\usepackage[utf8]{inputenc}


\begin{document}
 
%=========================================================

\title{Comparison of theorem provers}

\author{Artem Yushkovskiy
\\\textnormal{\texttt{artem.yushkovskiy@aalto.fi}}}

\affiliation{\textbf{Tutor}: Stavros Tripakis}

\maketitle

%==========================================================

\begin{abstract}
One of the useful applications mathematical logic theory is the Automated theorem proving. This is a set of techniques that allow one to verify mathematical statements mechanically using logical reasoning. Although, it can be used to solve engineering problems as well, for instance to prove security properties for a software system or an algorithm. Furthermore, automated theorem proving is an essential part of the Artificial Intelligence theory, which became highly evolving these days. In this paper, the bases of formal systems and automated deduction theory are described. Then, two widespread tools for automated theorem proving, Coq [1] and Isabelle [2], are compared with respect to expressive power and usability.

\vspace{3mm}
\noindent KEYWORDS: logic, automated theorem prover, Coq, Isabelle

\end{abstract}


%============================================================


\section{Introduction}

Modern world cannot be imagined without mathematics. Almost every human technical achievement is based on the groundwork of mathematical methods. Formal models allow one both to describe existent phenomena and to develop new tools.
Although in most cases formal models that one built work efficiently, in previous century mathematics experienced deep fundamental crisis caused by the need for a formal definition of the very bases of mathematics. That time many paradoxes in some mathematics have been discovered, so, in 1920s, the three main confrontation schools appeared:
the \textit{logicism} based on work of Gottlob Frege, Bertrand Russell, and Alfred Whitehead, which stated that mathematics is an extension of logic and therefore it can be reduced to logic partly or fully;
the \textit{formalism} led by David Hilbert, also called the 'proof theory', which advocated classical mathematics with its axiomatic approach;
and the radical \textit{intuitionism} led by Luitzen Brouwer, which rejected this formalisation as unnecessary and even meaningless, claiming that only the objects, which we know how to construct, may exist~[Fer08].
Although the formalism was the leading school, all these schools have contributed important ideas and techniques to mathematics, philosophy and many other scientific and engineering areas.

One solution for the foundational crisis was proposed by the school of formalism, it is known as Hilbert's program, and aimed to base all existing theories on finite set of axioms, and prove that these sets are consistent~[Zac06]. Thus Hilbert proposed to reduce the consistency of all mathematics to basic arithmetic. 
Unfortunately, these intentions turned out to be rather unrealisable, when in 1931 Kurt Gödel published his two famous incompleteness theorems, that demonstrated the limitations of any formal axiomatic system containing basic arithmetic. In particularly, he proved that all consistent axiomatic formulations of number theory are incomplete (such that there will always be statements which cannot be proved or disproved within that formulation), and that such formal system can not prove that itself it is consistent (i.e., it is able to prove that a statement and its negative are both true).
Notwithstanding the fact that the whole mathematics can not be described in a single axiomatic system, the formal approach allows one to build though restricted, but fine, concise and verifiable theories.

% // unnecessary
% The search for foundations of mathematics is a central question of the philosophy of mathematics; the abstract nature of mathematical objects presents special philosophical challenges.

% //from wiki:
% More generally, the foundations of mathematics can be described as the study of basic mathematical concepts (number, geometrical figure, set, function, etc.) and how they form hierarchies of more complex structures and concepts, especially the fundamentally important constructions that form the language of mathematics (formulae, theories and their models giving a meaning to formulae, definitions, proofs, algorithms, etc.) also called metamathematical concepts, with an eye to the philosophical aspects and the unity of mathematics. 

Informally, \textit{formal proof} is the sequence of statements, based on finite set of fundamental axioms and satisfying the rules of logical inference. \textit{The axiom} is a statement claimed to be true evidently. \textit{The logical inference} is the transfer from one statement (premise) to another (consequence), which preserve truth, while the rule of logical inference is a principle that allows one to infer the validity of such transfer. 

In formal logic, inference is based entirely on the structure (i.e., form) of those statements, which allows one to apply basic logical rules to any type of proof and thus construct the formal system.
The main goal of the formal system is to be verifiable, i.e. one could \textit{check} its validity. At present, a lot of tools are being developing to automate the process of such checking to run it on the computer. In particular, the systems \textit{Isabelle/ZF}~[???], \textit{Coq}~[???], PVS~[???], ACL~[???] work in a form of axiomatic set theory and allow the user to enter theorems and proofs into the computer, which then verifies that the proof is correct (these are also called sometimes 'proof assistants').
Another goal of constructing the formal system is having the computer to \textit{discover} formal proof. This goal is different from the previous one since the system must be optimised for efficient search. The output proofs can rely on induction, or on meta argument, or on higher-order logic. McCune’s systems \textit{Otter}~[???] and \textit{Prover9}~[???] are commonly recognized as the state-of-the-art tools~[Com00].

In current paper we consider only the systems, which are built to achieve the first goal, i.e. to verify existing proof, since <...>. Two aforementioned theorem provers Coq and Isabelle are examined for the purpose of revealing expressiveness, computation power and usability. These properties are described in detail in section <???>. Section~\ref{sec:formal_theory} gives an overview of the history of logic providing thorough definitions, typology and properties of formal system and formal proof. Issues related to theoretical limitations of formal systems are discussed further as well. <... about comparison, results and author's personal contrubution>

%============================================================

\section{Theory of logical calculi}
\label{sec:formal_theory}
// TODO

Basics: set theory, its problems (paradoxes), Zermelo–Fraenkel, propositional and 1st ordered logic (all with unified formal notation), formal languages, ...

% maybe Higher-Order Logic, Non-classical Logics - for interest. See https://plato.stanford.edu/entries/reasoning-automated/

MOCK from plato: % see https://plato.stanford.edu/entries/reasoning-automated/#DedCal
A third important consideration in the building of an automated reasoning program is the selection of the actual deduction calculus that will be used by the program to perform its inferences. As indicated before, the choice is highly dependent on the nature of the problem domain and there is a fair range of options available: General-purpose theorem proving and problem solving (first-order logic, simple type theory), program verification (first-order logic), distributed and concurrent systems (modal and temporal logics), program specification (intuitionistic logic), hardware verification (higher-order logic), logic programming (Horn logic), and so on.

MOCK from wiki: A formal system or logical calculus is any well-defined system of abstract thought based on the model of mathematics. A formal system need not be mathematical as such; for example, Spinoza's Ethics imitates the form of Euclid's Elements. Spinoza employed Euclidiean elements such as "axioms" or "primitive truths", rules of inferences etc. so that a calculus can be built using these. For nature of such primitive truths, one can consult Tarski's "Concept of truth for a formalized language".

Foundational crisis in mathematics (Intuitionism, Logicism, Formalism, ) Hilbert’s program

%------------------------------------------------------------

\subsection{Basic properties of logical system}
\label{sec:logic_properties}
// TODO

According to Formalists: Independence, Consistency, Completeness, Decideability

Formalists defined four basic properties which every logical system must have:
1. Independence, which means that there aren’t any superfluous axioms. There’s no axiom that can be derived from the other axioms.
2. Consistency, which means that no theorem of the system contradicts another.
3. Completeness, which means the ability to derive all true formulae from the axioms.
4. Decideability, the Entscheidungsproblem, which asks for an algorithm that takes as input a statement and answers "Yes" or "No" according to whether the statement is universally valid, i.e., valid in every structure satisfying the axioms.

%------------------------------------------------------------

\subsection{Limitations of logic systems}
\label{sec:limitations}
// TODO

completenes, Gödel's incompleteness theorems
% see: http://sakharov.net/foundation.html
% http://settheory.net/model-theory/completeness


%============================================================

\section{Methods for automated reasoning}
\label{sec:auto_reasoning}
// TODO

% from https://plato.stanford.edu/entries/reasoning-automated/#DedCal

techniques in common words (and in introduced previously notation): clause rewriting, 
- Resolution
- Sequent Deduction
- Natural Deduction
- The Matrix Connection Method
- Term Rewriting (+lambda calculus)
- Mathematical Induction

% perhaps separate into subsections


%============================================================

\section{Some automated theorem provers}
\label{sec:provers}
// TODO

in two words: here we consider Coq and Isabelle, bla-bla

%------------------------------------------------------------

\subsection{The Coq theorem prover}
\label{sec:prover_coq}

Describe Coq features

%------------------------------------------------------------

\subsection{The Isabelle theorem prover}
\label{sec:prover_isabelle}

Describe Isabelle features


%============================================================

\section{Comparison of selected theorem provers}
\label{sec:comparison}
// TODO

in table:

- expressiveness of logic used
- time of proving
- num of supporting theories
- set of techniques to prove automatically
- V of theorem code
- num of user interaction steps
- usability
- etc ...


%============================================================

\section{Results}
\label{sec:results}
// TODO

Conclusions of comparison

%============================================================

\section{Future work}
\label{sec:future_work}
// TODO

to apply this survey to software verification


%============================================================


\bibliographystyle{plain}
\bibliography{cs-seminar}

\end{document}
