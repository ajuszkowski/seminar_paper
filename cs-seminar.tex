\documentclass[article]{aaltoseries}
\usepackage[utf8]{inputenc}
\usepackage{enumitem}  % remove spaces before enumerations and lists
\usepackage{url}
\begin{document}

%=========================================================

\title{Comparison of theorem provers}

\author{Artem Yushkovskiy
\\\textnormal{\texttt{artem.yushkovskiy@aalto.fi}}}

\affiliation{\textbf{Tutor}: Stavros Tripakis}

\maketitle

%==========================================================

\begin{abstract}
One of the useful applications of mathematical logic theory is Automated theorem proving. This is a set of techniques that allow one to verify mathematical statements mechanically using logical reasoning. Although it can be used to solve engineering problems as well, for instance, to prove security properties for a software system or an algorithm.
% 2. Examples (e.g.) and explanations (i.e.) require commas directly before and after the abbreviations, as well as at the end of the examples or the explanation. Alternatively, you can use the following phrases: such as, including and for example.\\Examples\\A paramagnetic contrast agent, e.g., gadolinium-DTPA, affects the MR signal. \A paramagnetic contrast agent ( e.g., gadolinium-DTPA) ...\A paramagnetic contrast agent, such as gadolinium-DTPA, ....\A paramagnetic contrast agent, for example, gadolinium-DTPA, ...\A paramagnetic contrast agent, including gadolinium-DTPA, ...\\The entire electricity industry throughout the developed world, i.e., North America, Europe, Australia and Japan, has been in deep transformation since 1990s.
% 3. Use more formal time adverbials\\This adverbial phrase is a bit informal in style and is more suitable for textbooks and general writing. Below you find more typical expressions used in academic writing. Notice that adverbial phrases are usually separated from the subject with a comma.\\Recently,\In recent years,\\Until recently,\Currently,\Presently,\At present, etc.
% 4. Use as subject\\The sentence would flow better if this information/noun phrase was placed at the beginning of the sentence and used as a subject.
% 5., 6. Sentence / clause ends with a verb\\Avoid ending a clause or sentence with a verb. Reword this sentence to place the verb early in the sentence.
% TODO: probably drop out the sentence below ("i'm not sure in what sense theorem proving is part of AI, but i guess you'll describe this below")
Furthermore, automated theorem proving is an essential part of the Artificial Intelligence theory, which is highly evolving in recent years.
%continue...
This paper compares two widespread tools for automated theorem proving, Coq~\cite{tool_Coq} and Isabelle~\cite{tool_Isabelle}, with respect to the power of expressiveness and usability. For this reason, it firstly gives a brief introduction to the bases of formal systems and automated deduction theory, its main problems and challenges.

\vspace{3mm}
\noindent KEYWORDS: logic, formal method, proof theory, automated theorem prover, Coq, Isabelle.
	
\end{abstract}


%============================================================

\section{Introduction}

Modern world cannot be imagined without mathematics. Almost every human technical achievement is based on the groundwork of mathematical methods. Using such formal methods, one both to describe existent phenomena and to develop new tools in a verifiable way, using strict logical \textit{proofs}.
% 7. Avoid personal references\\Does your field of study allow you to use personal references (we, us, our, I, my) or spoken passive forms (people, they, you, one, someone)?\\Most academic fields avoid using personal references, such as I, my, our, we, or you, and especially such collective nouns as people, one, they, someone or somebody to replace the regular passive verb form.\\Personal references can be tricky, so it is important that you study research texts in your field to find out when you can or cannot use personal references that refer to the writer(s) , such as we, our, us, I and my. In contrast, no academic field appears to allow the use of you to refer to your reader.\\EXAMPLES\\Informal\One cannot improve efficiency without increasing the resources.\\Formal\\Efficiency cannot be improved without increasing the resources.
% 8. Singular plural error\\This noun should be changed either into singular or plural form.\\Examples\\Nouns modified by numbers are usually in the plural form.\\The project is divided into three parts.\Some nouns have only singular form, even though they are have a plural meaning:\\Wrong:\All equipments are configured automatically.\\Correct:\All equipment are configured automatically.\Similarly, the word data looks like it is singular; in fact, it is actually the plural form of the word datum.\\Be careful to not confuse uncountable words with those that can have plural forms. In the example below, research cannot have a plural form because it is an uncountable noun.\\Wrong:\Many researches have been conducted in recent years.\\Correct:\Much research has been conducted in recent years.
% // unnecessary
% The search for foundations of mathematics is a central question of the philosophy of mathematics; the abstract nature of mathematical objects presents special philosophical challenges.

% //from wiki:
% More generally, the foundations of mathematics can be described as the study of basic mathematical concepts (number, geometrical figure, set, function, etc.) and how they form hierarchies of more complex structures and concepts, especially the fundamentally important constructions that form the language of mathematics (formulae, theories and their models giving a meaning to formulae, definitions, proofs, algorithms, etc.) also called metamathematical concepts, with an eye to the philosophical aspects and the unity of mathematics. 

Informally, a \textit{formal proof} is the sequence of statements, based on a finite set of fundamental axioms and satisfying the rules of logical inference. \textit{The axiom} is a statement claimed to be true evidently. \textit{The logical inference} is the transfer from one statement (\textit{premise}) to another (\textit{consequence}), which preserve truth, while the rule of logical inference is a principle that allows one to infer the validity of such transfer.
% 9. Determiners: new or generic info --> a(n)\\This noun/noun phrase is new or unknown information and should signalled with the indefinite article a or an, if the noun is countable, or without an article, if the noun is uncountable.\\The indefinite article is also used in general references, i.e., the word refers to any member of a group.\\Examples\\I need a pen (Any pen will do as long as it works).\An mp3-player is a device ...
% 10. Partitive structure requires "a(n)" + of\\Uncountable and plural countable nouns postmodified with partitive of-phrases function as generics. Such forms require an indefinite article (a/an) or zero article.\\Examples\\A glass of water can save a life.\Smoking a pack of cigarettes a day is not unusual for heavy smokers.\This device complies with a set of industry standards.
In formal logic, inference is based entirely on the structure (i.e., form) of those statements, which allows one to apply basic logical rules to any type of proof and thus construct the formal system.

The main goal of the formal system is to be verifiable, i.e. one could \textit{check} its validity. At present, a lot of tools are being developing to automate the process of such checking to run it on the computer. In particular, the systems \textit{Isabelle/ZF}~\cite{tool_Isabelle}, \textit{Coq}~\cite{tool_Coq}, \textit{PVS}~\cite{tool_Pvs}, \textit{ACL2}~\cite{tool_Acl} work in a form of axiomatic set theory and allow the user to enter theorems and proofs into the computer, which then verifies that the proof is correct (these are also called sometimes 'proof assistants').
Another goal of constructing the formal system is having the computer to automatically \textit{discover} the formal proof, which can rely either on induction, or on meta argument, or on higher-order logic. McCune’s systems \textit{Otter}~\cite{tool_Otter} and \textit{Prover9}~\cite{tool_Prover9} are commonly recognized as the state-of-the-art tools.

In current paper we consider only the systems, which are built to achieve the first goal, i.e. to verify existing proof, since they can be sufficiently applied in software verification, the research area in which both authors are being involved. Two aforementioned theorem provers Coq and Isabelle are examined for the purpose of revealing expressiveness, computation power and usability. These properties are described in detail in Section~\ref{sec:comparison}. 
In Section~\ref{sec:formal_history}, one can find an overview of the history of logic. Then, Section~\ref{sec:formal_theory} gives thorough definitions, typology and basic properties of formal system and formal proof; issues related to theoretical limitations of formal systems are discussed there as well. General methods for automated reasoning are given in Section~\ref{sec:auto_reasoning}. Possible applications areas for automated theorem provers are enumerated in Section~\ref{sec:applications}. The comparison of selected theorem provers is given in Section~\ref{sec:comparison}.

<... results and author's personal contribution >


%============================================================

\section{History of formal approach}
\label{sec:formal_history}

Although in most cases our formal models work efficiently, in previous century mathematics experienced deep fundamental crisis caused by the need for a formal definition of the very basis of mathematics. That time many paradoxes in some mathematics have been discovered, so, in 1920s, the three main confrontation schools appeared:
the \textit{logicism} based on work of Gottlob Frege, Bertrand Russell, and Alfred Whitehead, which stated that mathematics is an extension of logic and therefore it can be reduced to logic partly or fully;
the \textit{formalism} led by David Hilbert, also called the 'proof theory', which advocated classical mathematics with its axiomatic approach;
and the radical \textit{intuitionism} led by Luitzen Brouwer, which rejected this formalisation as unnecessary and even meaningless, claiming that only the objects, which we know how to construct, may exist.
Although the formalism was the leading school, all these schools have contributed important ideas and techniques to mathematics, philosophy and many other scientific and engineering areas~\cite{Fer08}.

One solution for the foundational crisis was proposed by the school of formalism, it is known as Hilbert's program, and aimed to base all existing theories on finite set of axioms, and prove that these sets are consistent~\cite{Zac06}. Thus Hilbert proposed to reduce the consistency of all mathematics to basic arithmetic. 
Unfortunately, these intentions turned out to be rather unrealisable, when in 1931 Kurt Gödel published his two famous incompleteness theorems, that demonstrated the limitations of any formal axiomatic system containing basic arithmetic~\cite{Raa15}. In particularly, he proved that all consistent axiomatic formulations of number theory are incomplete (such that there will always be statements which cannot be proved or disproved within that formulation), and that such formal system can not prove that itself it is consistent (i.e., it is unable to prove that a statement and its negative are both true).
Notwithstanding the fact that the whole mathematics can not be described in a single axiomatic system, the formal approach allows one to build though restricted, but fine, concise and verifiable theories.

%============================================================

\section{Logical foundations}
\label{sec:formal_theory}

// TODO (Paragraph is still in progress)

% a very little banch of history
\begin{itemize}
\itemsep0em
	\item what the logical system is (formally: set of axioms, inference rules)
	\item (?) a brief history of axiomatic approach (Euclid, Hilbert)
	\item what the truth is: completeness
	\item intro to set theory (for notation) +Zermelo–Fraenkel set theory (ZFC)
	\item intro to formal languages
	\item intro to propositional and 1st ordered logic (introduce notation here. Maybe Higher-Order Logic, Non-classical Logics)
	\item type systems (+dependent type, where type checking although may be undecidable, but it verifies the correctness of ), Nominal vs. structural type system. \textbf{Curry–Howard correspondence} (the direct relationship between computer programs and mathematical proofs)
	\item else?
\end{itemize}

According to Formalists: Independence, Consistency, Completeness, Decidability.

Formalists defined four basic properties which every logical system must have:
\begin{enumerate}
\itemsep0em
	\item Independence, which means that there aren’t any superfluous axioms. There’s no axiom that can be derived from the other axioms.
	\item Consistency, which means that no theorem of the system contradicts another.
	\item Completeness, which means the ability to derive all true formulae from the axioms.
	\item Decidability, the Entscheidungsproblem, which asks for an algorithm that takes as input a statement and answers "Yes" or "No" according to whether the statement is universally valid, i.e., valid in every structure satisfying the axioms.
\end{enumerate}

Formula (i.e. statement, program) may be:
\begin{itemize}
\itemsep0em
	\item provable (either true or false, according to selected set of axioms)
	\item valid
	\item sound
	\item ...
\end{itemize}

%============================================================

\section{Methods for automated reasoning}
\label{sec:auto_reasoning}

// TODO (Paragraph is still in progress)

techniques in common words (and in introduced previously notation), e.g.: 
\begin{itemize}
\itemsep0em
	\item Clause rewriting
	\item Resolution
	\item Sequent Deduction
	\item Natural Deduction
	\item The Matrix Connection Method
	\item Term Rewriting (+lambda calculus)
	\item Mathematical Induction
\end{itemize}


% perhaps separate into subsections

%============================================================

\section{Some applications of theorem provers}
\label{sec:applications}

// TODO (Paragraph is still in progress) 

Describe possible applications of formal methods:
\begin{enumerate}
	\itemsep0em
	\item Interactive theorem proving: construct a formal axiomatic proof of correctness, 
	\item verifying that a mathematical statement is true.
	\item verifying that a circuit description, an algorithm, or a network or security protocol meets its specification:
	\begin{itemize}
		\item program verification (first-order logic), 
		\item distributed and concurrent systems (modal and temporal logics), 
		\item program specification (intuitionistic logic),
		\item Model checking: reduce to a finite state space, and test exhaustively.
		\item hardware verification (higher-order logic), 
		\item logic programming (Horn logic), 
		\item and so on.
	\end{itemize}
\end{enumerate}

%============================================================

\section{Comparison of some theorem provers}
\label{sec:comparison}

//TODO (Paragraph is still in progress) 

(in two words: here we consider Coq and Isabelle, bla-bla)

%MOCK from "Certified Programming with Dependent Types" by Adam Chlipala.
%/* ACL2 is notable in this field for having only a first-order language at its foundation. That is, you cannot work with functions over functions and all those other treats of functional programming. By giving up this facility, ACL2 can make broader assumptions about how well its proof automation will work, but we can generally recover the same advantages in other proof assistants when we happen to be programming in first-order fragments

%Isabelle/HOL and Coq both support coding new proof manipulations in ML in ways that cannot lead to the acceptance of invalid proofs. Additionally, Coq includes a domain-specific language for coding decision procedures in normal Coq source code, with no need to break out into ML.

%A language with dependent types may include references to programs inside of types. For instance, the type of an array might include a program expression giving the size of the array, making it possible to verify absence of out-of-bounds accesses statically. Dependent types can go even further than this, effectively capturing any correctness property in a type. PVS’s dependent types are much more general, but they are squeezed inside the single mechanism of subset types, where a normal type is refined by attaching a predicate over its elements. Each member of the subset type is an element of the base type that satisfies the predicate. 
%*/

%------------------------------------------------------------

\subsection{The Coq theorem prover}
\label{sec:prover_coq}

// TODO (Paragraph is still in progress)

Coq is a formal proof assistant system. It provides a formal language to write mathematical definitions, executable algorithms and theorems together with an environment for semi-interactive development of machine-checked proofs~\cite{tool_Coq}. Coq uses the Calculus of Construction, a higher-order formalism for constructive proofs in natural deduction style, developed by Thierry Coquand~\cite{Coq86}. 
The Calculus of Construction can be considered as an extension of the Curry–Howard isomorphism. % in a way that the latter associates a term in the simply typed lambda calculus with each natural-deduction proof in intuitionistic propositional logic, when the Calculus of Construction extends this isomorphism to proofs in the full intuitionistic predicate calculus, which includes proofs of quantified statements (which are also called "propositions").

(add some core features of Coq ...)

(Coq has been used to formalize ...)

%------------------------------------------------------------

\subsection{The Isabelle theorem prover}
\label{sec:prover_isabelle}

// TODO (Paragraph is still in progress)

The Isabelle is an interactive theorem prover, which relies on higher-order logic. It allows mathematical formulas to be expressed in a formal language and provides tools for proving those formulas in a logical calculus~\cite{tool_Isabelle}. Isabelle's main proof method is a higher-order version of resolution, based on higher-order unification.

(add some core features of Isabelle ...)

(Isabelle has been used to formalize ...)

%MOCK from wiki:
%/* it is based on a small logical core to ease logical correctness. Isabelle is generic: it provides a meta-logic (a weak type theory), which is used to encode object logics like first-order logic (FOL), higher-order logic (HOL) or Zermelo–Fraenkel set theory (ZFC). Isabelle's main proof method is a higher-order version of resolution, based on higher-order unification. Though interactive, Isabelle also features efficient automatic reasoning tools, such as a term rewriting engine and a tableaux prover, as well as various decision procedures. Isabelle has been used to formalize numerous theorems from mathematics and computer science, like Gödel's completeness theorem, Gödel's theorem about the consistency of the axiom of choice, the prime number theorem, correctness of security protocols, and properties of programming language semantics. The Isabelle theorem prover is free software, released under the revised BSD license.
% */

%------------------------------------------------------------

\subsection{Joint comparison}
\label{sec:joint_comparison}

//TODO: in table:
\begin{itemize}
	\itemsep0em
	\item expressiveness of logic used
	\item time of proving
	\item num of supporting theories
	\item set of techniques to prove automatically
	\item Volume of proof (as text)
	\item num of user interaction steps
	\item usability
	\item etc ...
\end{itemize}

%============================================================

\section{Results}
\label{sec:results}

// TODO (Paragraph is still in progress)
% Pr Tripakis' note: "i think this version can be merged into section 7, perhaps as its last subsection, 7.x"
conclusion of comparison

%============================================================

\section{Future work}
\label{sec:future_work}

// TODO (Paragraph is still in progress)

< in future, we want to apply this survey to software verification >


%============================================================


\bibliographystyle{ieeetr}
\bibliography{cs-seminar}

\end{document}
